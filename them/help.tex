\newpage
\subsection{Сводка необходимых формулировок с прошлого семестра.}

\begin{Def}
Пусть $\GG_s$ "--- КГ замкнутая жорданова кривая, ограничивающая область $D_s$ $(s=1,\dots,S),\\
S \ge 2$ "--- натуральное.
Так же пусть все $\ol{D_2},\dots,\ol{D_S}$ лежат внутри области $D_1$. \\
Тогда область $D = D_1 \diagdown \bigsqcup\limits_{s=1}^S \ol{D_s}$ называется \underline{допустимой областью ранга $S$}. \\(S "--- порядок связности области $D$)
\end{Def}

\begin{Lem}[Лемма Гурса (условие $\D$)]
Пусть $D$ "--- область в $\C$, $f \in C(D)$ тогда для $\forall$ замкнутого треугольника $\D \in D$ имеем  $\Gint{\partial^{+}\D} f(z)dz = 0$. В данном случае говорят, что функция $f$ удовлетворяет условию треугольника.  
\end{Lem}

\begin{The}[Интегральная теорема Коши для допустимой области]
Пусть $D$ "--- допустимая область с~границей $\partial^{+}D = \Gamma_{1}^{+} \bigsqcup \Gamma_{2}^{-}\bigsqcup \dots \bigsqcup \Gamma_{S}^{-} $, $f \in C(\ol{D})$, $f$ удовлетворяет условию $\D$ в $D$. Тогда $\Gint{\partial^{+}D} f(z)dz = 0$. 

\end{The}