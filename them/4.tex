\newpage
\section{Примеры вычисления интегралов. Преобразования Фурье и Гильберта.}
Для вычисления интгралов в смысле главного значения в первую очередь мы будем использовать теорему~о~вычетах а так же предложения (1.1) --- (1.4).

\begin{enumerate} 
\item $f(z) = \frac{\ln{z}}{z^2-1}$
\item Вспомним замечание (1.1).
Рассмотрим $f(z) = \frac{1}{z}$ обычный несобственный интеграл от $-\infty$ до $+\infty$
\[
\int\limits_{-\infty}^{+\infty} \frac{1}{z}dz \text{"--- расходится}
\]
Однако, интеграл в смысле главного значения сходится и
\[
(vp)\int\limits_{-\infty}^{+\infty} \frac{1}{z}dz = 0
\]
\item Преобразование Фурье
\[
  \hat f(x) :=\frac{1}{\sqrt{2 \pi}}(vp)\Gint{\R}f(y)\,e^{ixy}dy;\qquad
  \Til f(x) :=\frac{1}{\sqrt{2 \pi}}(vp)\Gint{\R}f(y)\,e^{ixy}dy.
\]
\item Преобразование Гильберта
\[
f \to H [f](x) = \frac{1}{\pi}(vp)\Gint{\R}\frac{f(x)}{x-t}dt
\]
\end{enumerate}