
 \subsection{Программа курса} 
 Программа курса «Комплексный анализ», часть II, 3 поток, механики (6 сем., 2014/2015 уч. год).
\begin{enumerate}
\item   Интеграл в смысле главного значения. Вычет относительно области и его   
      вычисление. 
\item Лемма Жордана.
\item   Теорема о вычетах для интеграла в смысле главного значения. 
\item   Примеры вычисления интегралов. Преобразования Фурье и Гильберта. 
\item  Теорема о логарифмических вычетах. Принцип аргумента. 
\item  Теорема Руше. Принцип сохранения области и его следствие.
\item  Конформность голоморфных инъективных функций.
\item  Обратный принцип соответствия границ.
\item   Критерии локальной однолистности и локальной обратимости.
\item Принцип симметрии Римана-Шварца для конформных отображений.
\item Теорема Римана о конформном отображении (б/д). Теорема Каратеодори 
      для жордановых областей (б/д). Их гидродинамическая интерпретация.
\item Обтекание цилиндра с вихрем. Поведение линий тока. Вполне 
      регулярность.
\item Обтекание профиля Жуковского. Условие Чаплыгина.
\item Уравнение Эйлера-Бернулли для идеальной жидкости. Формула 
      Чаплыгина для подъемной силы крыла.
\item Формула Жуковского для подъемной силы крыла.
\item Вычисление подъемной силы для профиля Жуковского.
\item Аналитическое продолжение вдоль пути и его свойства. 
\item Единственность аналитического продолжения вдоль пути и его связь с 
      продолжением по цепочке.
\item Гомотопные пути в области.  Связь 1- и 2- гомотопности путей в области. 
      Классы гомотопных замкнутых путей в  $\C \diagdown \{0\}$. Эквивалентные 
      определения односвязной области в  $\C$ (б/д).
\item Аналитическое продолжение по близким путям и по путям гомотопии. 
      Теорема о монодромии.
\item Аналитическое продолжение первообразной. Теорема об интегралах  по 
      гомотопным путям.
\item Полная аналитическая функция  (ПАФ)  в смысле Вейерштрасса. 
      Теорема Пуанкаре-Вольтерра. Голоморфные ветви и точки аналитичности  
      ветвей ПАФ. 
\item Точки ветвления (ветвей) ПАФ, их классификация. ПАФ $\Ln z$ и $z^p$  . 
\item Первообразная рациональной функции как ПАФ. ПАФ $Arctg z$ . 
\item Модулярная функция и малые теоремы Пикара. 

\end{enumerate}