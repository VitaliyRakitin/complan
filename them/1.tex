\newpage
\section{Интеграл в смысле главного значения. Вычет относительно области и его вычисление.}

\subsection{Интеграл в смысле главного значения.}
\begin{Def} 
$\g\colon[\a,\b]\to\C$ "--- \underline{кусочно-гладкий путь} (КГ-путь), если он гладкий на каждом из отрезков (т.е. $\exists \dot{\g}(t)$).
\end{Def}

\begin{Def} 
$\GG = \{\g \} $ "--- класс эквивалентности КГ-путей называется \underline{КГ-кривой}.
\end{Def}


\begin{Def} 
Пусть $\GG$ "--- КГ-кривая в $\ol\C$, а $[\GG]$ "--- траектория, $z_0 \not\in [\GG], z_0 \in \C$.\\ 
Если $\frac{1}{z-z_0}\circ\GG$ "--- КГ-кривая в $\C$, тогда $\GG$ "--- \underline{КГ-кривая в $\ol\C$}.
\end{Def}


\begin{Def} 
\underline{Допустимая кривая} "--- жарданова КГ-кривая в $\C$ (взаимооднозначная) либо замкнутая-жорданова.
\end{Def}

Теперь давайте определим интеграл в смысле главного значения. 
\begin{enumerate}
\item Пусть $\GG$ "--- допустимся кривая в $\ol\C$, $[\GG]$ "--- траектория $\GG$. 
\item Положим $\A = \{a_1,\dots,a_J\} \in [\GG]$ "--- конечное множество $\colon$ $f \in C([\GG] \diagdown \A)$ "--- комплекснозначная функция.
\item Для $\forall a_j$ $\exists \d_j \in (0,+\infty)$. Обозначим $\D = \{\d_1,\dots,\d_J\}$, а $|\D| = {\max_{j=1,\dots,J}} \{\d_j\}$.
\item Пусть $\exists \d > 0 \colon$  при $|\D| < \d$ круги $\{B(a_j,\d_j)\}_{j=0}^J$ "--- попарно не пересекаются, \\
причем~для $\forall j\colon$ $\partial B(a_j,\d_j) \cap [\GG]$ содержит не более 2 точек (ровно 2, если $\GG$ --- замкнуто).  
\item Обозначим через $\GG_{A\D} = \GG \diagdown \bigsqcup\limits_{j=1}^J B(a_j,\d_j)$ "--- цепь кривых (выкинули точки с радиусами).
\end{enumerate}
Тогда существует $I = (vp)\Gint{\GG} f(z)dz = \lim\limits_{|\D| \to \infty} \Gint{\GG_{A\D}} f(z)dz$ "--- главное значение интеграла.

\begin{Def} 
$I$ "--- называется \underline{интегралом в смысле главного значения}, если
% для любого $\e > 0$ \\существует $\d > 0\colon$ $|\D| < \d$ и $|I - \Gint{\GG_{A\D}} f(z)dz| < \e$.
\[
\text{ для } \forall \e >0 \  % пробел, который генегируется с помощью «\text{ }», можно поставить используя два символа: «\ »
\exists \d > 0  \colon |\D| < \d \text{ и } \bigg|I - \Gint{\GG_{A\D}} f(z)dz\bigg| < \e. % В случае интегралов и сумм размер скобок лучше подбирать руками
\]
\end{Def}

\begin{Zam}[от Белошапки]
Важно отметить, чем принципиально отличается интеграл в смысле главного значения от обычного интеграла.  Рассмотрим интеграл $\int\limits_{a}^{b} f(x) dx$  над вещественной прямой.\\ Пусть он имеет единственную особенность в точке $c \in [a,b]$. Тогда
\[
\int\limits_{a}^{b} f(x) dx = \lim\limits_{\d_1 \to 0} \lim\limits_{\d_2 \to 0}\left( \int\limits_{a}^{c-\d_1} f(x) dx + \int\limits_{c+\d_2}^{b} f(x) dx \right) .
\] 
Такой интеграл называется несобственным с особенностью в точке $c$. Однако интеграл в смысле главного значения определяется следующим образом
\[
(vp)\int\limits_{a}^{b} f(x) dx = \lim\limits_{\d \to 0}\left( \int\limits_{a}^{c-\d} f(x) dx + \int\limits_{c+\d}^{b} f(x) dx \right).
\]

Таким образом интеграл в смысле главного значения отличается от несобственного интеграла тем, что в случае главного значения приближаемся к особенности согласованно с обеех сторон. В свою очередь для несобственного интеграла сходимости справа и слева независимы друг от друга.
\end{Zam}


\subsection{Вычет относительно области.}
Пусть $D$ "--- допустимая область в $\ol\C$ ранга $s \ge 1$. \\
(т.е. для $\forall z_0 \in \C \diagdown \ol{D} \not= 0$ отображение $\frac{1}{z-z_0}$ переводит $D$ в некоторую обычную допустимую область в $\C$).

Граница области $D\colon$ $\partial^{+} D = \GG_1 \sqcup \Gamma_2^{-}\sqcup \dots \sqcup \Gamma_s^{-}.  $ 


Пусть для $\forall a \in \ol{D}$ $\exists \d_a > 0 \colon$ при всех $\d \in (0,\d_a)\colon$ $\partial B(a,\d) \cap \partial_{\ol\C} D$ содержит не более 2х точек и $\partial B(a,\d) \cap \ol{D}$ является связной замкнутой кривой, ориентированной против часовой стрелки, если $a \not= \infty$, и по часовой, если $a = \infty$. Обозначим её за $\g_s^{+}(a)$.


\begin{Def} 
Пусть в рамках вышеизложенных обозначений $\exists \d_a^{'} > 0 \colon f \in C(B'(a,\d_a^{'})\cap \ol{D})$.\\ Тогда, если при $\d < \min(\d_a,\d_a^{'})$ определён $\Gint{\g_{\d}^{'}(a)}f(z)dz$,   то  \underline{вычет в точке $a$ относительно области $D$}$\colon$
\[
\res\limits_{a,D} f(z) := \lim\limits_{\d \to 0} \frac{1}{2\pi i} \Gint{\g_{\d}^{'}(a)}f(z)dz.
\] 
\end{Def}


\subsection{Вычисление вычета относительно области.}
Положим $D$ "--- допустимая область в $\C$. Для вычисления вычетов относительно области используем следующие предложения

\begin{Pre}
Пусть $a \in D$ и $f \in A(U'(a))$ (проколотая окрестность), тогда $\res\limits_{a, D}f(z) = \res\limits_{a}f(z).$
\end{Pre}
\begin{proof}
Очевидно.
\end{proof}

\begin{Pre}
Пусть $a \in \C$ и $f(z) = \ol{\ol{o}}(\frac{1}{z-a})$ или $a =\infty$ и при $z \to \infty$ $(z \in D)\colon$  $f(z) = \ol{\ol{o}}(\frac{1}{z}) $, \\тогда $\res\limits_{a, D}f(z) = 0.$
\end{Pre}
\begin{proof}
\begin{enumerate}
\item Случай $a \not= \infty$. 
$\left|  \Gint{\gamma_{\d}^{+}} f(z)dz \right| \le \ol{\ol{o}}(\frac{1}{\d}) \cdot 2\pi\d \Dte 0$, где $\d = |z - a|$.
\item Случай $a = \infty$. ?
\end{enumerate}
\end{proof}


\begin{Pre}
Пусть $a \in \partial D$ "--- полюс первого порядка для функции $f$, $a \not= \infty$, $\Theta_a$ "--- абсолютная величина внутреннего угла области $D$ с вершиной $a$. Тогда
\[
\res\limits_{a, D}f(z) = \frac{\Theta_a}{2\pi}\res\limits_{a}f(z).
\]
\end{Pre}
\begin{proof} При достаточно малом $\d$ положим $\gamma_{\d}^{+}(a) =\partial^{+}B(a,\d) \cap \ol{D}$ "--- связная дуга $\d$-окрестности точки~a. Рассмотрим параметризацию данной кривой $\gamma_{\d}^{+}(a) \colon z = z(t) = a\d e^{it}$, $\dot z(t) = a\d i  e^{it}$,  при $t \in [\a(\d),\b(\d)]$, причем~$\a(\d)<\b(\d)<\a(\d)+2\pi$.
Положим
\[
\Theta_a = \lim\limits_{\d \to 0} (\b(\d) - \a(\d) ).
\]
Разложим $f$ в проколотой окрестности точки $a$ в ряд Лорана при $0 < |z-a| <\d_a >0 \colon$
\[
f(z) = \frac{C_{-1}}{z-a} + C_0 + \dots
\]
Тогда вычет в точке $a$ относительно области $D$ принимает вид
 \[
 \res\limits_{a,D} = \frac{1}{2 \pi i} \Gint{\gamma_{\d}^{+}} f(z)dz = \frac{1}{2\pi i} \lim\limits_{\d \to 0} \int\limits_{\a(\d)}^{\b(\d)} \left(\frac{C_{-1}}{\d e^{it}} + \ol{\ol{o}}(1)\right)\d ie^{it}dt = \frac{C_{-1}}{2\pi}\lim\limits_{\d \to 0} \left(\a(\d) - \b(\d)\right) = C_{-1}\frac{\Theta_a}{2\pi}
 \]
 $C_{-1}$ "--- и есть необходимый вычет.

\end{proof}

\begin{Pre}[Лемма Жордана] 
Пусть $f \in C(\Pi_R)$, где $\Pi_R = \{z \in \C |\Im{z} \ge 0,  |z| > R\}$, и для $z \in \Pi_R\colon$ $\lim\limits_{z \to \infty} f(z) = 0$, тогда для любого фиксированного $\l > 0$ выполняется соотношение 
\[
\res\limits_{\infty, D}(f(z)e^{i \l z}) = 0.
\] (подробнее смотри билет 2)
\end{Pre}
