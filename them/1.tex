\newpage
\section{Интеграл в смысле главного значения. Вычет относительно области и его вычисление.}

\subsection{Интеграл в смысле главного значения.}
\begin{Def} 
$\g\colon[\a,\b]\to\C$ "--- \underline{кусочно-гладкий путь} (КГ-путь), если он гладкий на каждом из отрезков (т.е. $\exists \dot{\g}(t)$).
\end{Def}

\begin{Def} 
$\GG = \{\g \} $ "--- класс эквивалентности КГ-путей называется \underline{КГ-кривой}.
\end{Def}


\begin{Def} 
Пусть $\GG$ "--- КГ-кривая в $\ol\C$, а $[\GG]$ "--- траектория, $z_0 \not\in [\GG], z_0 \in \C$.\\ 
Если $\frac{1}{z-z_0}\circ\GG$ "--- КГ-кривая в $\C$, тогда $\GG$ "--- \underline{КГ-кривая в $\ol\C$}.
\end{Def}


\begin{Def} 
\underline{Допустимая кривая} "--- жарданова КГ-кривая в $\C$ (взаимооднозначная) либо замкнутая-жордановая.
\end{Def}

Теперь давайте определим интеграл в смысле главного значения. 
\begin{enumerate}
\item Пусть $\GG$ "--- допустимся кривая в $\ol\C$, $[\GG]$ "--- траектория $\GG$. 
\item Положим $\A = \{a_1,\dots,a_J\} \in [\GG]$ "--- конечное множество $\colon$ $f \in C([\GG] \diagdown \A)$ "--- комплекснозначная функция.
\item Для $\forall a_j$ $\exists \d_j \in (0,+\infty)$. Обозначим $\D = \{\d_1,\dots,\d_J\}$, а $|\D| = {\max_{j=1,\dots,J}} \{\d_j\}$.
\item Пусть $\exists \d > 0 \colon$  при $|\D| < \d$ круги $\{B(a_j,\d_j)\}_{j=0}^J$ "--- попарно не пересекаются, \\
причем~для $\forall j\colon$ $\partial B(a_j,\d_j) \cap [\GG]$ содержит не более 2 точек (ровно 2, если $\GG$ --- замкнуто).  
\item Обозначим через $\GG_{A\D} = \GG \diagdown \bigsqcup\limits_{j=1}^J B(a_j,\d_j)$ "--- цепь кривых (выкинули точки с радиусами).
\end{enumerate}
Тогда существует $I = (vp)\Gint{\GG} f(z)dz = \lim\limits_{|\D| \to \infty} \Gint{\GG_{A\D}} f(z)dz$ "--- главное значение интеграла.

\begin{Def} 
$I$ "--- называется \underline{интегралом в смысле главного значения}, если
% для любого $\e > 0$ \\существует $\d > 0\colon$ $|\D| < \d$ и $|I - \Gint{\GG_{A\D}} f(z)dz| < \e$.
\[
\text{ для } \forall \e >0 \text{ } \exists \d > 0  \colon |\D| < \d \text{ и } \left|I - \Gint{\GG_{A\D}} f(z)dz\right| < \e.
\]
\end{Def}


\subsection{Вычет относительно области.}
Пусть $D$ "--- допустимая область в $\ol\C$ ранга $s \ge 1$. \\
(т.е. для $\forall z_0 \in \C \diagdown \ol{D} \not= 0$ отображение $\frac{1}{z-z_0}$ переводит $D$ в некоторую обычную допустимую область в $\C$).

Граница области $D\colon$ $\partial^{+} D = \GG_1 \sqcup \Gamma_2^{-}\sqcup \dots \sqcup \Gamma_s^{-}.  $ 


Пусть для $\forall a \in \ol{D}$ $\exists \d_a > 0 \colon$ при всех $\d \in (0,\d_a)\colon$ $\partial B(a,\d) \cap \partial_{\ol\C} D$ содержит не более 2х точек и $\partial B(a,\d) \cap \ol{D}$ является связной замкнутой кривой, ориентированной против часовой стрелки, если $a \not= \infty$, и по часовой, если $a = \infty$. Обозначим её за $\g_s^{+}(a)$.


\begin{Def} 
Пусть в рамках вышеизложенных обозначений $\exists \d_a^{'} > 0 \colon f \in C(B'(a,\d_a^{'})\cap \ol{D})$.\\ Тогда, если при $\d < \min(\d_a,\d_a^{'})$ определён $\Gint{\g_{\d}^{'}(a)}f(z)dz$,   то  \underline{вычет в точке $a$ относительно области $D$}$\colon$
\[
\res\limits_{a,D} f(z) := \lim\limits_{\d \to 0} \frac{1}{2\pi i} \Gint{\g_{\d}^{'}(a)}f(z)dz.
\] 
\end{Def}
